\chapter{Analisi di pareti interne e solai}
L'analisi è continuata prendendo in considerazione la parte strutturale delle pareti portanti interne e dei solai intermedi e di copertura.
Per quanto riguarda l'isolante e il rivestimento esterno, invece, l'analisi è stata decretata conclusa con i vincitori visti nel capitolo precedente.

Per analizzare pareti interne e solai si è scelto di procedere soltanto con il migliore di quelle esterne, ovvero l'\xlam, e di paragonarlo alla struttura zero.
Il pacchetto prevede quindi per le pareti portanti interne la stessa tipologia di quelle esterne, mentre per i solai (uguali tra interpiano e copertura) si è prevista un'omogeneità tra i materiali: cordoli in calcestruzzo e laterocemento per quando riguarda la soluzione zero, e pannelli in \xlam{} per quella lignea.

A differenza delle pareti esterne, per i solai non vi è alcun guadagno nell'utilizzare spessori diversi. 
Pertanto non è stata presa in considerazione la parte di superficie risparmiata. 
Non è stata presa in considerazione nemmeno per le pareti portanti interne in quanto (se ne veda la pianta) di piccola lunghezza e trascurabile.
Le altre pareti, quelle non strutturali, sono state infatti lasciate come da soluzione zero: in cartongesso.

\e stato possibile, tra l'altro, decretare la totale similitudine tra scavi e fondazioni delle due soluzioni.
Come si vede nella WBS, le fondazioni continue lungo le pareti murarie strutturali coincidono tra soluzione zero e \xlam.

Nell'analisi è quindi presente soltanto la componente del costo del materiale, nella quale si è utilizzato il prezzario provinciale.
Negli allegati di pagina \pageref{STRUTMuraturaTotaleIntESol} e successiva è presente il computo metrico estimativo, mentre in tabella \ref{STRUTConfrontoFinale} ne è presentato un riassunto confrontandolo con le parti analizzate prima.
\begin{table}[htb]
\caption[Confronto dei costi dell'intera struttura]{Confronto dei costi dell'intera struttura paragonando i costi totali e i guadagni dovuti alla vendita della superficie risparmiata}
\label{STRUTConfrontoFinale}
\centering
\begin{tabular}{@{}lrrrr@{}}
\toprule
 & \multicolumn{1}{c}{Muratura portante} & \multicolumn{1}{c}{X-LAM} & \multicolumn{2}{c}{Differenza} \\
 & \multicolumn{1}{c}{\teuro} & \multicolumn{1}{c}{\teuro} & \multicolumn{2}{c}{\teuro} \\\midrule
Pareti esterne & 51.330,52 & 94.881,44 & 43.550,92 & \multirow{2}{*}{-13.089,08} \\
Guadagno par esterne & 0,00 & -56.640,00 & -56.640,00 &  \\
Pareti interne e solai & 210.892,97 & 351.564,99 & & 140.672,01   \\
 &  &  &  &  \\\midrule
Totale & 262.223,49 & 389.806,43 & & 127.582,93   \\ \bottomrule
\end{tabular}
\end{table}

Si può notare come il costo del materiale in \xlam{} sia così maggiore per pareti interne e solai, a tal punto da riuscire a superare il guadagno che si era accumulato nella precedente analisi.
E per tanto a decretarne il maggior costo finale di questa soluzione.

Nel prossimo paragrafo si cercherà di spiegarne i motivi di tale costo del materiale maggiore.
Si elencheranno anche i motivi grazie ai quali, benché il costo maggiore, si possa o meno propendere per tale scelta.








