\section{Rivestimento esterno}
\begin{table}[htbp]
\caption{Elenco dei rivestimenti esterni presi in considerazione e descrizione sintetica delle loro caratteristiche. Per quelli con un asterisco $^\star$ è stata effettuata un'analisi dei prezzi}
\label{MaterialiRIV}
\centering
\begin{tabularx}{\textwidth}{rXX}
    \toprule
        \textbf{A} & \textbf{BenessereBio intonaco $\,^\star$} & Biointonaco termo-deumidificante, antimuffa e anti condensa. 
        Adatto a tutti i tipi di muro. Traspirante ed ad alta efficienza energetica. \\\midrule
        \textbf{B} & \textbf{Intomap R1 $\,^\star$} & Intonaco di fondo su murature miste, laterizio nuovo, blocchi in calcestruzzo e cemento armato gettato.  Buona adesione, particolarmente indicato per essere applicato con intonacatrice. 
        Rasatura con Planitop 510 in calce-cemento a tessitura fine.\\\midrule
        \textbf{C} & \textbf{Powerpanel HD $\,^\star$} & Rivestimento per ambienti esterni con lastre in conglomerato cementizio, di peso ridotto, facili da lavorare e durevoli agli agenti atmosferici.\\\midrule
        \textbf{D} & \textbf{Facciata ventilata BBuilding in larice} & Costituita da pannelli premontati in stabilimento composti da una sottostruttura portante e da doghe posate orizzontalmente, completamente realizzata in legno di larice massello non impregnato.\\\midrule
        \textbf{E} & \textbf{Facciata ventilata gres porcellanato} & Sistema strutturale di rivestimento esterno degli edifici per combinare estetica, funzionalità, manutenzione ed efficienza energetica. \\\midrule
        \textbf{F} & \textbf{Facciata ventilata in lamiera Prefa} & Possono essere utilizzate sia per esterni che per interni per rivestire pareti e soffitti. Il fissaggio a scomparsa mediante un sistema ad incastro maschio - femmina garantisce un'estetica gradevole. Si ottengono facciate di facile manutenzione e all'avanguardia per molti decenni.\\
    \bottomrule
\end{tabularx}
\end{table}
\begin{landscape}
\begin{table}[p]
\caption{Piano di manutenzione del rivestimento esterno. 
\e stato considerato rispetto i 30 anni di vita utile del migliore, ovvero la facciata ventilata in lamiera Prefa.}
\label{RIV_PianoManutenzione}
\centering\scriptsize
\begin{tabular}{@{}lllllrcr@{}}
\toprule
Elem. Tecnico & Tipologia Elem. & Tipologia & Soggeto & Cadenza & Costo singolo & Ripetizioni & \multicolumn{1}{l}{Costo} \\ 
Manutenibile & Tecnico & intervento & Esecutore & (Anni) & intervento (\teuro) & \multicolumn{1}{l}{previste} & \multicolumn{1}{c}{totale (\teuro)} \\ \midrule
\multirow{4}{*}{Intonaco} & \multirow{4}{*}{BenessereBio intonaco} & Controllo generale delle parti a vista & Utente & 0,5 &   0,00 & 60 &   0,00 \\
 &  & Pulizia delle superfici & Pittore & 10 &   6.608,00 & 2 &   13.216,00 \\
 &  & Sostituzione delle parti più soggette ad usura & Muratore & 10 &   891,09 & 2 &   1.782,18 \\
 &  &  &  &  &  & TOT &   14.998,18 \\ \midrule
\multirow{4}{*}{Intonaco} & \multirow{4}{*}{Intomap R1} & Controllo generale delle parti a vista & Utente & 0,5 &   0,00 & 60 &   0,00 \\
 &  & Pulizia delle superfici & Pittore & 10 &   6.608,00 & 2 &   13.216,00 \\
 &  & Sostituzione delle parti più soggette ad usura & Muratore & 10 &   995,83 & 2 &   1.991,65 \\
 &  &  &  &  &  & TOT &   15.207,65 \\ \midrule
\multirow{4}{*}{Rivestimento lapideo} & \multirow{4}{*}{Powerpanel HD} & Controllo generale delle parti a vista & Utente & 1 &   0,00 & 30 &   0,00 \\
 &  & Pulizia delle superfici con idropulitrice & Pittore & 10 &   5.286,40 & 2 &   10.572,80 \\
 &  & Sostituzione delle parti più soggette ad usura & Pavimentista & 10 &   1.984,05 & 2 &   3.968,10 \\
 &  &  &  &  &  & TOT &   14.540,90 \\\midrule
\multirow{4}{*}{Rivestimento ligneo} & \multirow{4}{*}{Facciata ventilata BBuilding} & Controllo generale delle parti a vista & Utente & 0,5 &   0,00 & 60 &   0,00 \\
 &  & Ripristino degli strati protettivi & Pittore & 10 &   7.486,86 & 2 &   14.973,73 \\
 &  & Sostituzione delle parti più soggette ad usura & Pavimentista & 10 &   5.088,16 & 2 &   10.176,32 \\
 &  &  &  &  &  & TOT &   25.150,05 \\ \midrule
\multirow{4}{*}{Rivestimento lapideo} & \multirow{4}{*}{Facciata ventilata gres} & Controllo generale delle parti a vista & Utente & 0,5 &   0,00 & 60 &   0,00 \\
 &  & Ripristino degli strati protettivi & Pittore & 10 &   2.808,40 & 2 &   5.616,80 \\
 &  & Sostituzione delle parti più soggette ad usura & Pavimentista & 10 &   2.940,56 & 2 &   5.881,12 \\
 &  &  &  &  &  & TOT &   11.497,92 \\\midrule
\multirow{2}{*}{Rivestimento metallico} & \multirow{2}{*}{Facciata ventilata lamiera} & Controllo generale delleparti a vista & Utente & 0,5 &   0,00 & 60 &   0,00 \\
 &  &  &  &  &  & TOT &   0,00 \\ \bottomrule
\end{tabular}
\end{table}
\end{landscape}
Per quanto riguarda il rivestimento esterno, non per tutti è stato possibile trovare direttamente il prezzo unitario nel prezzario. 
Pertanto, per quelli che in tabella \ref{MaterialiRIV} riportano un asterisco, è stato necessario eseguire un'analisi dei prezzi utilizzando i dati dichiarati dai produttori. 
Questo a causa della particolare natura di quel materiale e specifico di un'azienda in particolare o relativamente nuovo sul mercato da esser aggiunto al prezzario della Provincia di Trento.

Nel caso del rivestimento esterno è stata valutata come molto incidente la manutenzione da attuare in corso d'opera, essendo il rivestimento esposto alle intemperie e non protetto come gli altri strati. 
Non è stato preso in considerazione il tempo di posa, perché si è supposto fosse simile tra tutti.
Nemmeno il parametro dello spessore è stato considerato, perché in questo caso sono tutti ben che minimo uguali.

A causa quindi della durata del materiale è stato creato un piano manutenzione esposto nella tabella \ref{RIV_PianoManutenzione} a pagina \pageref{RIV_PianoManutenzione}.

Nella tabella \ref{RIVvincitore} è riportato il sommario tra prezzi unitari finali e la manutenzione. 
Sono riportati infine i costi finali del rivestimento, ed evidenziando come la soluzione con l'intonaco A è quella più conveniente in termini di costi totali.
\begin{table}[htb]
\caption[Analisi del rivestimento esterno]{Analisi del rivestimento esterno. Somma tra i costi del materiale e i costi di manutenzione. L'ultima colonna evidenza il minor costo finale della soluzione A.}
\label{RIVvincitore}
\centering\scriptsize
\begin{tabular}{@{}rrrrr@{}}
\toprule
& \multicolumn{1}{c}{Prezzo unitario} & \multicolumn{1}{c}{Costo materiale} & \multicolumn{1}{c}{Manutenzione} & \multicolumn{1}{c}{Costo}  \\ 
 & \multicolumn{1}{c}{\teuro/mq} & \multicolumn{1}{c}{\teuro} & \multicolumn{1}{c}{\teuro} & \multicolumn{1}{c}{\teuro} \\\midrule
A & 14,97 &  9.893,09 &  14.998,18 &  \cellcolor[HTML]{3FE52C}24.891,27 \\
B & 18,14 &  11.986,91 &  15.207,65 &  \cellcolor[HTML]{13AE14}27.194,56 \\
C & 46,05 &  30.429,84 &  14.540,90 &  \cellcolor[HTML]{CFE703}44.970,74 \\
D & 140,00 &  92.512,00 &  25.150,05 &  \cellcolor[HTML]{F66E51}117.662,05 \\
E & 75,00 &  49.560,00 &  11.497,92 &  \cellcolor[HTML]{FBDA59}61.057,92 \\
F & 130,00 &  85.904,00 &  0,00 &  \cellcolor[HTML]{FB813F}85.904,00 \\ \bottomrule
\end{tabular}
\end{table}
