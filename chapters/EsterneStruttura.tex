\section{Struttura}
Per quanto riguarda la struttura sono stati scelte diverse tipologie di soluzioni edilizie. 
Alcune di esse sono a muratura portante (caso coincidente con la soluzione zero di partenza) e altre presentano un telaio. 
Per queste ultime è stato calcolato anche la quota parte del tamponamento e non solo la parte strutturale. 
Come criteri di paragone sono stati presi in considerazione, oltre al costo del materiale, anche il tempo di posa e lo spessore minore. 
Si è valutato come molto incidente soprattutto il primo, perché, come si vedrà nelle considerazioni finali, il tempo di costruzione può cambiare radicalmente tra una soluzione costruita in opera ed una prefabbricata.
\begin{table}[H]
\caption{Elenco dei materiali strutturali presi in considerazione e descrizione sintetica delle loro caratteristiche.}
\centering
\begin{tabularx}{\textwidth}{rXX}
    \toprule
        \textbf{A} & \textbf{Muratura portante in laterizio} & Pannello rigido in lana di roccia non rivestito a doppia densità. 
        Adatto sia a isolamento termico che acustico, con ottimo comportamento al fuoco. 
        Ottima stabilità dimensionale e prestazionale. \\\midrule
        \textbf{B} & \textbf{Telaio in calcestruzzo e tamponamento in laterizio} & Struttura del telaio realizzata con pilastri in calcestruzzo armato classe XC0 di spessore $30\times\SI{30}{\centi\meter}$ e da travi di analoga tipologia con una dimensione $50\times\SI{30}{\centi\meter}$. Tamponamento realizzato con laterizi alveolati di spessore \SI{12}{\centi\meter}. \\\midrule
        \textbf{C} & \textbf{Telaio in acciaio e tamponamento in laterizio} & struttura realizzata con pilastri HEA200 e travi IPE120, con muratura in tamponamento con tavolato in laterizio alveolato da \SI{12}{\centi\meter}\\\midrule
        \textbf{D} & \textbf{Telaio in legno con chiusura in fibrogesso} & Pareti a telaio in montanti e traversi di legno massello in abete. Le pareti sono costituite da montanti e traversi di sezione 12 per \SI{8}{\centi\meter} disposti ad interasse tra 55 e \SI{65}{\centi\meter}, giuntati con apposita ferramenta metallica, strutturalmente controventate nel loro piano con una doppia lastra in fibrogesso. \\\midrule
        \textbf{E} & \textbf{X-LAM} & Pannelli strutturali di legno multistrato X-LAM in legno di abete. Pannelli in 5 strati per uno spessore totale di \SI{10}{\centi\meter}. Ditta produttrice KLH.\\
    \bottomrule
\end{tabularx}
\end{table}
Nello scegliere gli spessori di queste soluzioni si è cercato di dare una sensatezza strutturale e un'uniformità generale.

Tra gli allegati disponibili a fine della relazione, in particolare a pagina \pageref{STRUTcostoMateriale}, è riportato il computo metrico estimativo relativo a questa parte.

\begin{landscape}
\begin{table}[p]
\caption[Analisi costi della manodopera della parte strutturale esterna]{Analisi costi della manodopera della parte strutturale esterna. Dati presi dal tempario \cite{grosso2007tempario}. Per quanto riguarda la soluzione D il costo della manodopera è stato direttamente inserito in termini percentuali (il \SI{10}{\percent}) all'interno del costo totale finale. I dati dell'X-LAM della soluzione E sono stati ottenuti ri-elaborando dei computi metrici estimativi effettuati dall'azienda WoodCape SRL}
\label{STRUTtempario}
\centering\scriptsize
\begin{tabular}{llllrrrrrr}
\toprule
 &  Codice & Descrizione & u.m. & \multicolumn{1}{l}{Quant. ore} & \multicolumn{1}{l}{Quantità} & \multicolumn{1}{l}{Ore} & \multicolumn{1}{l}{Costo operaio} & \multicolumn{1}{l}{Totale (\teuro)} & \multicolumn{1}{l}{Costo totale}\\
  &   &  &  & \multicolumn{1}{l}{in cent. d'ora} & \multicolumn{1}{c}{(h)} & \multicolumn{1}{l}{} & \multicolumn{1}{l}{per ora (\teuro / h)} & \multicolumn{1}{l}{} & \multicolumn{1}{l}{manodopera (\teuro)}\\\midrule
\multirow{4}{*}{A} & TOC.130.330.f & Muratura monostrato in laterizio 30 cm & m2 & 0,6 & 198,24 & 118,94 & 30,99 & 3686,07 & \multirow{4}{*}{5.228,96} \\
 & TOC.100.110 & Conglomerato (cordolo) & m3 & 0,3 & 8,50 & 2,55 & 30,99 & 78,99 &  \\
 & TOC.100.300.c & Casseforme orizzontale cordolo & m2 & 0,48 & 84,96 & 40,78 & 30,99 & 1.263,80 &  \\
 & TOC.100.170.b & Armatura cordolo & kg & 0,019 & 339,84 & 6,46 & 30,99 & 200,10 &  \\\midrule
\multirow{5}{*}{B} & TOC.100.110 & Conglomerato (pilastri e travi) & m3 & 0,3 & 107,88 & 32,36 & 30,99 & 1.002,96 & \multirow{5}{*}{28.050,00} \\
 & TOC.100.220 & Casseforme pilastri & m2 & 0,48 & 369,60 & 177,41 & 30,99 & 5.497,87 &  \\
 & TOC.100.250 & Casseforme travi + sostegni & m2 & 0,96 & 293,92 & 282,16 & 30,99 & 8.744,24 &  \\
 & TOC.100.170.b & Armatura pilastri e travi & kg & 0,019 & 5.156,40 & 97,97 & 30,99 & 3.036,14 &  \\
 & TOC.130.330.c & Laterizi tamponamento 15 cm & m2 & 0,52 & 606,20 & 315,22 & 30,99 & 9.768,79 &  \\\midrule
\multirow{2}{*}{C} & TOC.160.110.b & pilastri e travi in acciaio & kg & 36,72 & 15.807,28 & 395,18 & 36,97 & 14.609,88 & 24.378,67 \\
 & TOC.130.330.c & Laterizi tamponamento 15 cm & m2 & 60,88 & 606,20 & 315,22 & 30,99 & 9.768,79 &  \\\midrule
D & \multicolumn{9}{c}{---} \\\midrule
E & Da azienda woodcape & Pareti strutturali perimetrali (15\teuro/m2) & m2 &  & 660,80 &  & 36,97 &  & 9.912,00\\\bottomrule
\end{tabular}
\end{table}
\end{landscape}

\begin{table}[tb]
\caption[Analisi della struttura perimetrale]{Analisi della struttura perimetrale in cui si somma il costo del materiale, il costo di posa in opera e la superficie guadagnata (Sg) vendendo lo spazio disponibile grazie allo spessore minore secondi i parametri riportati all'inizio di questo capitolo. Con differenza relativa si intende il costo totale rapportato rispetto alla soluzione zero A.}
\label{STRUTvincitore}
\centering\scriptsize
\begin{tabular}{lrrrrrrr}
\toprule
\multicolumn{1}{c}{} & \multicolumn{1}{c}{Costo materiale} & \multicolumn{1}{c}{Costo tempario} & \multicolumn{1}{c}{Spessore} & \multicolumn{1}{c}{Sup. guad. (Sg)} & \multicolumn{1}{c}{Vendita Sg} &  \multicolumn{1}{c}{Costo Tot.} & \multicolumn{1}{c}{Diff. relativa} \\
\multicolumn{1}{c}{} & \multicolumn{1}{c}{\teuro} & \multicolumn{1}{c}{\teuro} & \multicolumn{1}{c}{m} & \multicolumn{1}{c}{mq} & \multicolumn{1}{c}{\teuro} &  \multicolumn{1}{c}{\teuro} & \multicolumn{1}{c}{\teuro} \\\midrule
A & 46.102 & 5.229 & 0,30 & 0,00 & 0,000 & \cellcolor[HTML]{DAD06E}51.331 & \cellcolor[HTML]{DAD06E}0,000 \\
B & 99.640 & 28.050 & 0,30 & 0,00 & 0,000 & \cellcolor[HTML]{E67C73}127.690 & \cellcolor[HTML]{E67C73}-76.359 \\
C & 92.857 & 24.379 & 0,20 & 9,44 & 28.320 & \cellcolor[HTML]{F3AB6C}88.916 & \cellcolor[HTML]{F3AB6C}-37.585 \\
D & 96.322 & 9.634 & 0,12 & 16,99 & 50.976 & \cellcolor[HTML]{FFD666}54.980 & \cellcolor[HTML]{FFD666}-3.649 \\
E & 84.969 & 9.912 & 0,10 & 18,88 & 56.640 & \cellcolor[HTML]{57BB8A}38.241 & \cellcolor[HTML]{57BB8A}13.089
\\\bottomrule
\end{tabular}
\end{table}
