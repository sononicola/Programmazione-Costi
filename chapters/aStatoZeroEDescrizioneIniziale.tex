\chapter{Introduzione dell'analisi svolta}
Con questa relazione si riporta un'analisi economica di un edificio scelto, nella quale si è cercato di trovare i materiali da costruzione economicamente più vantaggiosi per la sua costruzione.
\section{Stato zero edificio scelto}
Come prima cosa si è scelto un edificio nel quale si avesse una certa quantità di documentazione da riuscire a trarne un computo metrico degli elementi.
\e stato quindi utilizzato un progetto di cinque villette a schiera, il quale, per l'occasione, è stato collocato in una specifica zona di Trento, in modo da poter ricavare i valori immobiliari di quella zona. 
Pertanto è stata scelta la zona $C5$ come è riportato nelle pagine \pageref{Edificio} e successive.

Sono stati scelti dei vincoli obbligatori da dover rispettare, inderogabili a seconda delle varie ipotesi successive. 
Tali vincoli rappresentano la dimensione tra gli interassi delle villette a schiera -- non doveva variare in base ai materiali scelti -- e la trasmittanza termica delle pareti esterne. 

Il progetto di partenza, riportato negli allegati a pagina \pageref{piante}, prevedeva un pacchetto della muratura esterna composto da: parete portante in laterizio, isolante in lana di roccia e intonaco esterno generico. Come solaio intermedio e di copertura una struttura in laterocemento.
Queste caratteristiche dei materiali compongono lo stato zero dell'edificio, e sono state usate come paragone rispetto gli altri materiali.

Questo stesso progetto è stato scomposto utilizzando la Tabella 21 delle \emph{OmniClass 2012}. 
Tale operazione, riportata nella WBS a pagina \pageref{WBS}, è servita per capire quali elementi variassero in base ai materiali scelti.
Le componenti uguali tra le varie opzioni, come si vedrà, non sono state analizzate in quanto non producevano differenze di costi.
\section{Descrizione del procedimento utilizzato per l'analisi}
Per valutare quali materiali fossero più vantaggiosi per la costruzione delle cinque villette a schiera, si è suddiviso il problema in più parti.
Si è partiti analizzando le parete perimetrali dell'intero edificio e definendo una stratigrafia della parete composta da tre macro categorie: struttura, isolante e rivestimento esterno.
Per queste tre categorie sono state fatte varie ipotesi di possibili materiali e per ciascuna di esse è stato trovato quello migliore, valutandone diversi aspetti. 
Per ogni materiale è stato valutato il costo ad opera conclusa utilizzando il prezzario della Provincia di Trento e -- dove ce ne fosse bisogno -- facendo un'analisi dei prezzi utilizzando i dati messi a disposizione dalle aziende produttrici.
In base alla tipologia del materiale sono stati presi in considerazione vari parametri come lo spessore, la manutenzione o la tempistica di messa in opera.

Si è continuata l'analisi concentrandosi solo sulla  struttura e valutando solamente le pareti interne, il solaio intermedio e quello di copertura.
Sono stati presi in considerazione il materiale vincente della precedente fase e quello della struttura zero di partenza.
Ovviamente si è fatto in modo che pareti e solai avessero una certa continuità e fossero quindi compatibili tra loro.

Come ultima fase sono state fatte delle considerazioni sui due materiali per cercare di trovare una motivazione a propendere per uno o l'altro, nonostante i costi maggiori. 

Nei prossimi capitoli verrà spiegato nel dettaglio l'analisi in ogni sua parte e verranno riportate le relative tabelle di confronto con i rispettivi computi.

Per non appesantire troppo il testo, alcune tabelle e alcuni documenti di contorno, sono stati riportati in fondo a questa relazione e citati, quando necessario, all'interno del testo.