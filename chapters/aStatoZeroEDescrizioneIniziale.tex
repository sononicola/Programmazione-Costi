\chapter{Introduzione dell'analisi svolta}
L'analisi è consistita nel trovare i materiali da costruzione economicamente più vantaggiosi per la costruzione di un edificio. 
\section{Stato zero edificio scelto}
balblablalbalallblalblbbla
\section{Descrizione del procedimento utilizzato per l'analisi}
Per valutare quali materiali fossero più vantaggiosi per la costruzione delle cinque villette a schiera, si è suddiviso il problema in più parti.
Si è partiti analizzando le parete perimetrali dell'intero edificio e definendo una stratigrafia della parete composta da tre macro categorie: struttura, isolante e rivestimento esterno.
Per queste tre categorie sono state fatte varie ipotesi di possibili materiali e per ciascuna di esse è stato trovato quello migliore, valutandone diversi aspetti. 
Per ogni materiale sono stati fatti dei computi metrici estimativi utilizzando il prezzario della Provincia di Trento e -- dove ce ne fosse bisogno -- facendo un'analisi dei prezzi utilizzando i dati messi a disposizione dalle aziende produttrici.
In base alla tipologia del materiale sono stati presi in considerazione vari parametri come lo spessore, la manutenzione o la tempistica di messa in opera.

Si è continuata l'analisi valutando le pareti interne, il solaio intermedio e di copertura per la sola macro categoria struttura. 
Sono stati presi in considerazioni come materiali il vincitore della precedente fase e quello della struttura zero di partenza.
Ovviamente si è fatto in modo che pareti esterne e solai avessero una continuità di materiale e fossero quindi compatibili tra loro.

Come ultima fase sono state fatte delle considerazioni sui due materiali per cercare di trovare una motivazione a propendere per uno o l'altro, nonostante i costi maggiori. 

Nei prossimi capitoli verrà spiegato nel dettaglio ogni sua parte dell'analisi e verranno riportate le relative tabelle di confronto e i computi.

Per non appesantire con troppi dati questa relazione, si riportano alla fine di essa, le piante della soluzione zero e le WBS create.
%
%Comando per le wbs e gli altri pdf:
%\includepdf[pages={1},pagecommand={\thispagestyle{plain}}]{img/RIV_AnalisiValore.pdf}