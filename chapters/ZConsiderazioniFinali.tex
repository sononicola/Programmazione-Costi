\chapter{Considerazioni sui finalisti}
Si vuole ora analizzare più del dettaglio il confronto tra la soluzione zero in muratura portante e la struttura in X-LAM. 
Si è visto come l'X-LAM sia più vantaggioso nel caso si analizzano solamente le pareti esterne, perché grazie al suo spessore minore (a parità di prestazioni) permette di risparmiare molta superficie e quindi di rivenderla.
Questo però non è sufficiente a colmare il maggior costo intrinseco del materiale non appena si prende in considerazione l'intera struttura dell'edificio. 
Nei solai infatti, non c'è nessun guadagno dall'avere uno spessore minore, se non quello di avere un pacchetto meno spesso, ma ciò non si traduce in ricavi.

Si elencheranno ora delle opzioni per poter capire se abbia senso o meno, a fronte dei \SI{180000}{\teuro} in più, scegliere comunque la soluzione in X-LAM. 
O se invece è preferibile optare per la muratura portante perché i benefici non sono abbastanza.
\paragraph{Tempi di costruzione}
Si è visto nel paragrafo dedicato alle strutture esterne, come il tempo sia stato molto incidente nel definire gli importi delle varie soluzioni adottate.
Non si è ancora considerato il cantiere nel suo intero insieme, considerando i tempi di attesa e di gestione.

Utilizzare un materiale che necessita di posa in opera, o uno che invece ha la sola necessità di essere assemblato, cambia radicalmente i tempi del cantiere.
Sebbene con l'X-LAM serva una gru per la posa dei pannelli, la posa e l'indurimento dei componenti in calcestruzzo del solaio in latero-cemento comporta un tempo che è dell'ordine di 1 o 2 mesi contro le 2-3 settimane necessarie con l'X-LAM. 
Avere un tempo di costruzione di alcune settimane in meno comporta sia la vendita anticipata: cosa che in alcune situazioni può significare enormi guadagni per il committente, che può iniziare ad avere un ritorno economico del proprio investimento in tempi molto più rapidi.
Sia costi di gestione del cantiere e di facilità di verifica della corretta posa enormemente minori. Avendo elementi prefabbricati realizzati tramite apposite macchine a taglio numerico CNC, tutti gli elementi hanno una perfetta dimensione rispetto il progetto. Non si avranno quindi possibilità di richiesta di varianti in corso d'opera dovute ad elementi che non si incastrano a causa di qualche centimetro diverso.

Per grandi strutture è differente anche il numero di operai che servono a realizzare la struttura. Utilizzando una struttura in X-LAM sono sufficienti tre persone per una normale unità familiare. Per ottenere lo stesso quantitativo di tempo utilizzando la muratura portante, sono necessarie almeno il doppio degli operai. Avendo così più costi di manodopera e più personale da riuscire a gestire all'interno del cantiere. 

Per riportare un esempio: il palazzo progettato da Land Lease a Meolbourne, ovvero il primo edificio di dieci piani ad essere costruito in X-LAM, è stato realizzato in appena 38 giorni. Comparandolo con un edificio simile costruito con telaio in calcestruzzo o con muratura portante, il tempo sale a 20 settimane \cite[39]{10storey}. 


\paragraph{X-LAM come isolante}
Nell'analisi del pacchetto isolante non si è considerato il fatto che l'X-LAM è ben più isolante rispetto la muratura portante. I due, infatti, hanno rispettivamente una conducibilità termica $\lambda$ di \SI{0.12}{} e di \SI{0.30}{W\per\metre K}.
Ciò consente di supporre che l'X-LAM potrebbe essere considerato come contribuente alla resistenza termica di parete. 

Mantenendo una resistenza voluta di \SI{4}{\square\metre K\per W} è possibile utilizzare \SI{2}{\centi\metre} in meno di isolante e ciò comporta due cose: si risparmiano altri due centimetri lungo tutto il perimetro, potendo ricavarne dalla vendita; si ha uno spessore isolante (utilizzando il vincitore D) ridotto a \SI{8}{\centi\metre} e che ha un prezzo unitario di \SI{23.73}{\teuro} al posto dei \SI{29.00}{\teuro} iniziali.

Questo si traduce in \SI{11370}{\teuro} recuperati dalla vendita e da \SI{3482.65}{\teuro} utilizzando l'isolante meno spesso.

Non solo. 
Se si considera l'intero \textit{Life Cycle Analysis} (LCA), avere molto isolante può comportare danni ambientali e di salubrità dell'ambiente interno \cite{reijnders_comprehensiveness_1999}. 
Avere un materiale che può in parte sostituirlo, sicuramente può aiutare. 

Un ulteriore considerazione va fatta sulla traspirabilità del materiale. 
Il legno infatti permette di avere un ambiente interno senza la presenza di muffe causate dall'umidità che è già presente nella malta e nell'intonaco in fase di posa.
\paragraph{X-LAM come legno}
Utilizzare un prodotto eco-sostenibile come il legno comporta innumerevoli vantaggi dal punto di vista della salute \cite{EnergyCost}. Dal punto di vista ambientale l'X-LAM è in grado di assorbire \SI{2314}{\tonne \per \cubic\metre} \cite{hammond2011inventory}.

Inoltre in termini economici si possono ottenere finanziamenti o mutui per la costruzione con determinati materiali. 

Utilizzando prodotti lignei è inoltre possibile acquisire vari punti extra nelle certificazioni energetiche quali Casa Clima o LEED. 
Certificazioni che potrebbero far aumentare non poco il valore immobiliare degli edifici costruiti. 

Un altro aspetto a vantaggio può essere l'opportunità di recuperare dei punti CAM a discapito di altri, in punti in cui magari, per certe situazioni, erano impossibili da rispettare.

Con uno sguardo più ampio: utilizzare prodotti lignei aiuta l'intera catena di gestione delle foreste \cite{callegari2010production}. 
Essendo una metodologia di costruzione nata prettamente in Italia e Austria contribuisce alla valorizzazione dell'industrializzazione nel nostro territorio..
\paragraph{Materiale leggero e performante}
L'X-LAM ha una densità circa la metà di quella della parete muraria. 
In più con i suoi \SI{400}{\kilogram\per\cubic\metre} riesce ad avere un rapporto resistenza-peso che è circa il doppio. 
Questo vuol dire che costruendo con l'X-LAM si ottiene una struttura quattro volte meno pesante e con la stessa resistenza. 
Si è visto infatti come non ci siano problemi a raggiungere edifici di dieci piani \cite{10storey}.

Avere una struttura leggera e duttile ha enormi benefici dal punto di vista sismico e permette di avere minori carichi trasmessi alle fondazioni. 
Il tutto si traduce in minori costi.

Un'ultima considerazione da fare riguarda la capacità di resistenza al fuoco. 
Sebbene in legno, la parte interna non soggetta al fuoco è capace di resistere al pare delle proprie capacità. 
Pertanto creando elementi nei quali si condiera questo aspetto, si ha un tempo di resistenza ben maggiore dell'acciaio presente nei solai in latero cemento (o meglio ancora dei telai in acciaio). 
Tutto questo senza l'utilizzo di vernici o materiali protettivi.


\subparagraph 
\newline
\enquote{CLT is cost-competitive because it already has thermal insulation, \omissis{} and for sure it might be a little bit more expensive in the beginning, but when you also include the maintenance costs it turns out be absulutely cost-competitive. -- \textcite{mallo_outlook_2014}}