\documentclass[a4paper,11pt]{report}
\usepackage{geometry}\geometry{a4paper,top=3.5cm,bottom=3.5cm,%
left=2.5cm,right=2.5cm,heightrounded,bindingoffset=0mm}
\usepackage[T1]{fontenc}
\usepackage[utf8]{inputenc}
\usepackage[italian]{babel}
\usepackage{graphicx}
\usepackage[table,xcdraw]{xcolor}
\usepackage{subfig}
\usepackage{amsmath,amsfonts,amssymb,braket,mathrsfs,eurosym}
\usepackage{float}
\usepackage{tabularx,booktabs,multirow}
\usepackage{lscape}
\usepackage{longtable}
\usepackage[output-decimal-marker={,}]{siunitx}
\usepackage{pdfpages}
%\usepackage{minipage}
\usepackage{tikz}
\usepackage{xspace}% per lo spazio intelligente
\newcommand{\e}{\`E\xspace}  %E'
\newcommand{\teuro}{\text{\euro}}%il simbolo dell'euro dentro le formula matematiche diventa "e". con \text diventa €
\newcommand{\red}[1]{\textcolor{red}{#1}}
\newcommand{\omissis}{[\textellipsis\unkern]} %puntini di sospensione
\usepackage{quoting}
\quotingsetup{font=small}
\usepackage{caption}
\captionsetup{tableposition=top,figureposition=bottom,font=small}\captionsetup{format=hang,labelfont={bf}}
\usepackage{plain}
\usepackage{titlesec} % per formato custom dei titoli dei capitoli
\usepackage{hyperref}
%\usepackage[style=numeric-comp]{biblatex}
\usepackage[autostyle,italian=guillemets]{csquotes}
%numeric: la bibliografia ha i numeri
%backref: indica nella bibliografia le pagine in cui ci sono le citazioni
\usepackage[style=numeric,backend=biber,backref,hyperref]{biblatex}
\addbibresource{Bibliografia.bib}
%\usepackage{appendix}
\begin{document}
\tableofcontents
%Tabelle e figure sulla stessa pagina:
%Le aggiunge all'indice. phantomsection serve per non far casini con hyperref
\clearpage
\begingroup
   %\let\cleardoublepage\relax  % book
    \let\clearpage\relax        % report
        \listoftables
        \phantomsection
        \addcontentsline{toc}{chapter}{Elenco delle tabelle}
        %
        \listoffigures
        \phantomsection
        \addcontentsline{toc}{chapter}{Elenco delle figure}
\endgroup
%\thispagestyle{empty}
% redefinizione del formato del titolo del capitolo
      % da formato
      %   Capitolo X
      %   Titolo capitolo
      % a formato
      %   X   Titolo capitolo 
	\titleformat{\chapter}
        {\normalfont\Huge\bfseries}{\thechapter}{1em}{}
	\titlespacing*{\chapter}{0pt}{0in}{0.02in}
	\titlespacing*{\section}{0pt}{0.2in}{0.02in}
	\titlespacing*{\subsection}{0pt}{0.10in}{0.02in}
%%%%%%%%%%%%%%%%%%%%%%%%%%%%%%%%%%%%%%%%%%%%%%%%%%%%%%%%%%%%
%%%%%%%%%%%%%%%%%%%%%%%%%%%%%%%%%%%%%%%%%%%%%%%%%%%%%%%%%%%%
%    
%bibliografia
%\nocite{mcgraw}%senza un preciso punto nel testo
%\bibliographystyle{plain}
\cleardoublepage
\phantomsection
\printbibliography[heading=bibintoc]
%Capitoli introduttivi
    \chapter{Introduzione dell'analisi svolta}
L'analisi è consistita nel trovare i materiali da costruzione economicamente più vantaggiosi per la costruzione di un edificio. 
\section{Stato zero edificio scelto}
balblablalbalallblalblbbla
\section{Descrizione del procedimento utilizzato per l'analisi}
Per valutare quali materiali fossero più vantaggiosi per la costruzione delle cinque villette a schiera, si è suddiviso il problema in più parti.
Si è partiti analizzando le parete perimetrali dell'intero edificio e definendo una stratigrafia della parete composta da tre macro categorie: struttura, isolante e rivestimento esterno.
Per queste tre categorie sono state fatte varie ipotesi di possibili materiali e per ciascuna di esse è stato trovato quello migliore, valutandone diversi aspetti. 
Per ogni materiale sono stati fatti dei computi metrici estimativi utilizzando il prezzario della Provincia di Trento e -- dove ce ne fosse bisogno -- facendo un'analisi dei prezzi utilizzando i dati messi a disposizione dalle aziende produttrici.
In base alla tipologia del materiale sono stati presi in considerazione vari parametri come lo spessore, la manutenzione o la tempistica di messa in opera.

Si è continuata l'analisi valutando le pareti interne, il solaio intermedio e di copertura per la sola macro categoria struttura. 
Sono stati presi in considerazioni come materiali il vincitore della precedente fase e quello della struttura zero di partenza.
Ovviamente si è fatto in modo che pareti esterne e solai avessero una continuità di materiale e fossero quindi compatibili tra loro.

Come ultima fase sono state fatte delle considerazioni sui due materiali per cercare di trovare una motivazione a propendere per uno o l'altro, nonostante i costi maggiori. 

Nei prossimi capitoli verrà spiegato nel dettaglio ogni sua parte dell'analisi e verranno riportate le relative tabelle di confronto e i computi.

Per non appesantire con troppi dati questa relazione, si riportano alla fine di essa, le piante della soluzione zero e le WBS create.
%
%Comando per le wbs e gli altri pdf:
%\includepdf[pages={1},pagecommand={\thispagestyle{plain}}]{img/RIV_AnalisiValore.pdf}
%Capitoli analisi pareti esterne
    \chapter{Analisi delle pareti esterne}
Sono stati scelti vari materiali e che verranno riportati nei paragrafi successivi con un'apposita descrizione.  
Come si vedrà in seguito, per ogni categoria sono stati valutati aspetti diversi, in quanto in alcuni casi diventavano del tutto trascurabili e alla pari tra i contendenti.

Come prima cosa è stata fatta un'analisi del valore per tutti i materiali in modo da riuscire ad avere un termine di confronto tra i materiali con diversi costi ma con proprietà molto diverse. 
Nell'analisi del valore riportata a pagina \pageref{fig:AnalisiValore} sono stati utilizzati i criteri che vengono riportati nella norma \textsc{UNI 829-2:1983}.
Per ogni criterio è stato assegnato un punteggio da 1 a 5  e utilizzando le caratteristiche riportate dalle aziende produttrici o dall'esperienza. 
Infine si è trovato il totale sommandoli. 
Come risulterà dalle analisi che verranno ora riportate, i maggiori punteggi ottenuti da ciascun materiale, corrispondono anche ai maggiori costi di essi.
Per alcuni risulterà talmente maggiore da non consentire nemmeno di poter fare un'analisi più approfondita.

Si elencano ora i diversi criteri che sono stati tenuti in conto.
Per tutte le categorie è stato calcolato il costo del materiale, preso da prezzario o da analisi prezzi.
A seconda dell'incidenza sono state prese come termine di paragone tempo di posa, manutenzione e spessore risparmiato.
Per quanto riguarda il tempo di posa i dati sono stati ottenuti dal tempario \textcite{grosso2007tempario} e convertiti in euro utilizzando il prezzo unitario dell'operaio (comune o specializzato). 
Nel caso del rivestimento è stato necessario elaborare un piano di manutenzione, come si vedrà in seguito. 
In questo caso è stato valutato come molto incidente i costi che si avranno successivamente alla costruzione dell'edificio.
Lo spessore minore di alcuni strati è stato utilizzato per calcolare la superficie libera risparmiata rispetto alla soluzione zero. 
Si è supposto poi di poter recuperare dei soldi da tale superficie vendendola o affittandola.
Sono stati presi i valori analizzando i prezzi di mercato forniti dall'Agenzia delle Entrate a Trento e ottenendo così un  prezzo di vendita di \SI{3000}{\teuro / \square\metre} e di di affitto di \SI{10.41}{\teuro /\square\metre mese }.


\begin{figure}[p]
    \centering 
    \includegraphics[width=0.9\textwidth]{img/AnalisiValore.pdf}
    \caption[Analisi del valore]{%
\scriptsize
\textbf{Affidabilità:} Capacità di mantenere sensibilmente invariata nel tempo la propria qualità in condizioni d'uso determinate
\textbf{Anigroscopicità:} Attitudine a non subire mutamenti di aspetto e/o morfologia, di dimensione e comportamento in seguito ad assorbimento di acqua o di vapor d'acqua
\textbf{Comodità d'uso e manovra:} Attitudine a presentare opportune caratteristiche di funzionalità, di facilità d'uso, di manovrabilità
\textbf{Controllo della combustione:} Realizzazione e mantenimento di condizioni tali da produrre processi di combustione a massimo rendimento di trasformazione e minima produzione di scorie e sostanze inquinanti
\textbf{Controllo della condensazione interstiziale:} Attitudine ad evitare la formazione di acqua di condensa all'interno degli elementi
\textbf{Controllo dell'inerzia termica:} Attitudine ad attenuare entro opportuni valori l'ampiezza di oscillazione della temperatura e a ritardarne di una opportuna entità l'effetto
\textbf{Controllo delle dispersioni di calore:} Contenimento entro determinati livelli delle perdite di calore per conduzione, convezione e irraggiamento
\textbf{Efficienza:} Capacità costante di rendimento nel funzionamento
\textbf{Facilità di intervento:} Possibilità di operare ispezioni, manutenzione e ripristini in modo agevole
\textbf{Idrorepellenza:} Attitudine a non essere penetrato da fluidi liquidi
\textbf{Integrazione:} Attitudine alla connessione funzionale e dimensionale
\textbf{Isolamento acustico:} Attitudine a fornire un'adeguata resistenza al passaggio di rumori
\textbf{Manutenibilità:} Possibilità di conformità a condizioni prestabilite entro un dato periodo di tempo in cui è compiuta l'azione di manutenzione
\textbf{Pulibilità:} Attitudine a consentire la rimozione di sporcizia e sostanze indesiderate
\textbf{Reazione al fuoco:} Grado di partecipazione di un materiale combustibile ad un fuoco al quale è sottoposto
\textbf{Recuperabilità:} Attitudine alla riutilizzazione di materiali o di elementi tecnici dopo demolizione o rimozione
\textbf{Resistenza al fuoco:} Attitudine a conservare, entro limiti determinati, per un intervallo di tempo determinato, le prestazioni fornite
\textbf{Riparabilità:} Attitudine a ripristinare l'integrità, la funzionalità e l'efficienza di parti o oggetti guasti 
\textbf{Sostituibilità:} Attitudine a consentire la collocazione di elementi tecnici al posto di altri}
\label{fig:AnalisiValore}
\end{figure}
        \section{Struttura}
Per quanto riguarda la struttura sono stati scelte diverse tipologie di soluzioni edilizie. 
Alcune di esse sono a muratura portante (caso coincidente con la soluzione zero di partenza) e altre presentano un telaio. 
Per queste ultime è stato calcolato anche la quota parte del tamponamento e non solo la parte strutturale. 
Come criteri di paragone sono stati presi in considerazione, oltre al costo del materiale, anche il tempo di posa e lo spessore minore. 
Si è valutato come molto incidente soprattutto il primo, perché, come si vedrà nelle considerazioni finali, il tempo di costruzione può cambiare radicalmente tra una soluzione costruita in opera ed una prefabbricata.
\begin{table}[H]
\caption{Elenco dei materiali strutturali presi in considerazione e descrizione sintetica delle loro caratteristiche.}
\centering
\begin{tabularx}{\textwidth}{rXX}
    \toprule
        \textbf{A} & \textbf{Muratura portante in laterizio} & Pannello rigido in lana di roccia non rivestito a doppia densità. 
        Adatto sia a isolamento termico che acustico, con ottimo comportamento al fuoco. 
        Ottima stabilità dimensionale e prestazionale. \\\midrule
        \textbf{B} & \textbf{Telaio in calcestruzzo e tamponamento in laterizio} & Struttura del telaio realizzata con pilastri in calcestruzzo armato classe XC0 di spessore $30\times\SI{30}{\centi\meter}$ e da travi di analoga tipologia con una dimensione $50\times\SI{30}{\centi\meter}$. Tamponamento realizzato con laterizi alveolati di spessore \SI{12}{\centi\meter}. \\\midrule
        \textbf{C} & \textbf{Telaio in acciaio e tamponamento in laterizio} & struttura realizzata con pilastri HEA200 e travi IPE120, con muratura in tamponamento con tavolato in laterizio alveolato da \SI{12}{\centi\meter}\\\midrule
        \textbf{D} & \textbf{Telaio in legno con chiusura in fibrogesso} & Pareti a telaio in montanti e traversi di legno massello in abete. Le pareti sono costituite da montanti e traversi di sezione 12 per \SI{8}{\centi\meter} disposti ad interasse tra 55 e \SI{65}{\centi\meter}, giuntati con apposita ferramenta metallica, strutturalmente controventate nel loro piano con una doppia lastra in fibrogesso. \\\midrule
        \textbf{E} & \textbf{X-LAM} & Pannelli strutturali di legno multistrato X-LAM in legno di abete. Pannelli in 5 strati per uno spessore totale di \SI{10}{\centi\meter}. Ditta produttrice KLH.\\
    \bottomrule
\end{tabularx}
\end{table}
Nello scegliere gli spessori di queste soluzioni si è cercato di dare una sensatezza strutturale e un'uniformità generale.

Tra gli allegati disponibili a fine della relazione, in particolare a pagina \pageref{STRUTcostoMateriale}, è riportato il computo metrico estimativo relativo a questa parte.

\begin{landscape}
\begin{table}[p]
\caption[Analisi costi della manodopera della parte strutturale esterna]{Analisi costi della manodopera della parte strutturale esterna. Dati presi dal tempario \cite{grosso2007tempario}. Per quanto riguarda la soluzione D il costo della manodopera è stato direttamente inserito in termini percentuali all'interno del costo totale finale. I dati dell'X-LAM della soluzione E sono stati ottenuti ri-elaborando dei computi metrici estimativi effettuati dall'azienda WoodCape SRL}
\label{STRUTtempario}
\centering\scriptsize
\begin{tabular}{llllrrrrrr}
\toprule
 &  Codice & Descrizione & u.m. & \multicolumn{1}{l}{Quant. ore} & \multicolumn{1}{l}{Quantità} & \multicolumn{1}{l}{Ore} & \multicolumn{1}{l}{Costo operaio} & \multicolumn{1}{l}{Totale (\teuro)} & \multicolumn{1}{l}{Costo totale}\\
  &   &  &  & \multicolumn{1}{l}{in cent. d'ora} & \multicolumn{1}{c}{(h)} & \multicolumn{1}{l}{} & \multicolumn{1}{l}{per ora (\teuro / h)} & \multicolumn{1}{l}{} & \multicolumn{1}{l}{manodopera (\teuro)}\\\midrule
\multirow{4}{*}{A} & TOC.130.330.f & Muratura monostrato in laterizio 30 cm & m2 & 0,6 & 198,24 & 118,94 & 30,99 & 3686,07 & \multirow{4}{*}{5.228,96} \\
 & TOC.100.110 & Conglomerato (cordolo) & m3 & 0,3 & 8,50 & 2,55 & 30,99 & 78,99 &  \\
 & TOC.100.300.c & Casseforme orizzontale cordolo & m2 & 0,48 & 84,96 & 40,78 & 30,99 & 1.263,80 &  \\
 & TOC.100.170.b & Armatura cordolo & kg & 0,019 & 339,84 & 6,46 & 30,99 & 200,10 &  \\\midrule
\multirow{5}{*}{B} & TOC.100.110 & Conglomerato (pilastri e travi) & m3 & 0,3 & 107,88 & 32,36 & 30,99 & 1.002,96 & \multirow{5}{*}{28.050,00} \\
 & TOC.100.220 & Casseforme pilastri & m2 & 0,48 & 369,60 & 177,41 & 30,99 & 5.497,87 &  \\
 & TOC.100.250 & Casseforme travi + sostegni & m2 & 0,96 & 293,92 & 282,16 & 30,99 & 8.744,24 &  \\
 & TOC.100.170.b & Armatura pilastri e travi & kg & 0,019 & 5.156,40 & 97,97 & 30,99 & 3.036,14 &  \\
 & TOC.130.330.c & Laterizi tamponamento 15 cm & m2 & 0,52 & 606,20 & 315,22 & 30,99 & 9.768,79 &  \\\midrule
\multirow{2}{*}{C} & TOC.160.110.b & pilastri e travi in acciaio & kg & 36,72 & 15.807,28 & 395,18 & 36,97 & 14.609,88 & 24.378,67 \\
 & TOC.130.330.c & Laterizi tamponamento 15 cm & m2 & 60,88 & 606,20 & 315,22 & 30,99 & 9.768,79 &  \\\midrule
D & \multicolumn{9}{c}{---} \\\midrule
E & Da azienda woodcape & Pareti strutturali perimetrali (15\teuro/m2) & m2 &  & 660,80 &  & 36,97 &  & 9.912,00\\\bottomrule
\end{tabular}
\end{table}
\end{landscape}

\begin{table}[tb]
\caption[Analisi della struttura perimetrale]{Analisi della struttura perimetrale in cui si somma il costo del materiale, il costo di posa in opera e la superficie guadagnata (Sg) vendendo lo spazio disponibile grazie allo spessore minore secondi i parametri riportati all'inizio di questo capitolo. Con differenza relativa si intende il costo totale rapportato rispetto alla soluzione zero A.}
\label{STRUTvincitore}
\centering\scriptsize
\begin{tabular}{lrrrrrrr}
\toprule
\multicolumn{1}{c}{} & \multicolumn{1}{c}{Costo materiale} & \multicolumn{1}{c}{Costo tempario} & \multicolumn{1}{c}{Spessore} & \multicolumn{1}{c}{Sup. guad. (Sg)} & \multicolumn{1}{c}{Vendita Sg} &  \multicolumn{1}{c}{Costo Tot.} & \multicolumn{1}{c}{Diff. relativa} \\
\multicolumn{1}{c}{} & \multicolumn{1}{c}{\teuro} & \multicolumn{1}{c}{\teuro} & \multicolumn{1}{c}{m} & \multicolumn{1}{c}{mq} & \multicolumn{1}{c}{\teuro} &  \multicolumn{1}{c}{\teuro} & \multicolumn{1}{c}{\teuro} \\\midrule
A & 46.102 & 5.229 & 0,30 & 0,00 & 0,000 & \cellcolor[HTML]{DAD06E}51.331 & \cellcolor[HTML]{DAD06E}0,000 \\
B & 99.640 & 28.050 & 0,30 & 0,00 & 0,000 & \cellcolor[HTML]{E67C73}127.690 & \cellcolor[HTML]{E67C73}-76.359 \\
C & 92.857 & 24.379 & 0,20 & 9,44 & 28.320 & \cellcolor[HTML]{F3AB6C}88.916 & \cellcolor[HTML]{F3AB6C}-37.585 \\
D & 96.322 & 9.634 & 0,12 & 16,99 & 50.976 & \cellcolor[HTML]{FFD666}54.980 & \cellcolor[HTML]{FFD666}-3.649 \\
E & 84.969 & 9.912 & 0,10 & 18,88 & 56.640 & \cellcolor[HTML]{57BB8A}38.241 & \cellcolor[HTML]{57BB8A}13.089
\\\bottomrule
\end{tabular}
\end{table}

        \section{Isolante}
\begin{table}[htbp]
\caption{Elenco degli isolanti presi in considerazione e descrizione sintetica delle loro caratteristiche.}
\label{MaterialiISO}
\centering
\begin{tabularx}{\textwidth}{rXX}
    \toprule
        \textbf{A} & \textbf{Rockwool VENTIROCK Duo} & Pannello rigido in lana di roccia non rivestito a doppia densità. 
        Adatto sia a isolamento termico che acustico, con ottimo comportamento al fuoco. 
        Ottima stabilità dimensionale e prestazionale. \\\midrule
        \textbf{B} & \textbf{Ecofine AEROGEL-HP} & Pannello termoisolante in aerogel con matrice di supporto in fibra minerale, ignifugo, permeabile al vapore, senza rivestimento. 
        Ottimo comportamento al fuoco. 
        Indicato in generale in tutte le applicazioni in cui si desideri o si sia vincolati a contenere lo spessore. O nel caso di ponti termici localizzati.\\\midrule
        \textbf{C} & \textbf{UNILIN uTherm PIR K} & Panello isolante PIR ad alte prestazioni con un rivestimento multistrato di carta metallizzata su entrambi i lati. 
        Molto leggero e facile da posare.\\\midrule
        \textbf{D} & \textbf{Stiferite Class SK} & Pannello sandwich costituito da un componente isolante in schiuma Polyiso, rivestito su entrambe le facce con velo vetro saturato.\\
    \bottomrule
\end{tabularx}
\end{table}
Passando ora all'analisi degli isolanti riportati in tabella \ref{MaterialiISO}, per trovare quello più conveniente, è stato fissato come parametro la resistenza termica dell'isolante a spessore maggiore, ovvero l'isolante A spesso \SI{14}{\centi\metre}.
\[R_A = \frac{s}{\lambda}=\frac{\SI{0.14}{m}}{\SI{0.035}{W/mK}}=\SI{4}{m^2K\per W}\]
Con questo parametro sono stati trovati di conseguenza gli spessori delle altre soluzioni che permettessero di equiparare (tenendo in considerazione gli spessori commerciali) la resistenza termica. 
Ottenendo così i dati di tabella \ref{tab:Isolanti}.
\begin{table}[htb]
\centering
\caption[Confronto degli spessori degli isolanti a parità di resistenza termica]{Spessori isolanti a parità di resistenza termica voluta pari a \SI{4}{m^2K/W}. Il prezzo unitario si riferisce al costo del solo materiale, senza trasporti e installazione.}
\label{tab:Isolanti}
\begin{tabular}{@{}rrrr@{}}
\toprule
  & Conducibilità & Spessore & Prezzo unitario \\ 
  & W/mK          & m        & euro/mq            \\ \midrule
A & 0,035         & 0,14     & 19,00           \\
B & 0,015         & 0,06     & 362,00          \\
C & 0,022         & 0,10     & 51,00           \\
D & 0,025         & 0,10     & 29,00           \\ \bottomrule
\end{tabular}%
\end{table}

Per decretare quale avesse il costo maggiore è stata presa in considerazione la superficie risparmiata grazie allo spessore minore degli isolanti B, C, D rispetto alla soluzione zero A.
\begin{table}[htb]
\caption{Analisi considerando il costo del materiale e il guadagno vendendo o affittando la superficie guadagnata grazie allo spessore ridotto di quell'isolante rispetto alla soluzione peggiore A. 
Dove con ogni riga si intende 1, 2, o 10 piani.
Evidenziati sono i guadagni o le perdite nel caso di vendita e utlizzando una superficie di due piani.}
\label{ISOvincitore}
\centering\scriptsize
\begin{tabular}{@{}crrrrrrrr@{}}
\toprule
& \multicolumn{1}{c}{Cost.Mat.} & \multicolumn{1}{c}{Diff. A} & \multicolumn{1}{c}{Sup. risp.} & \multicolumn{1}{c}{Vendita} &\multicolumn{1}{c}{Aff. 1} & \multicolumn{1}{c}{Aff. 10} & \multicolumn{1}{c}{Guad. Ven.} & \multicolumn{1}{c}{Guad. Aff.}  \\ 
& \multicolumn{1}{c}{\teuro} & \multicolumn{1}{c}{\teuro} & \multicolumn{1}{c}{\SI{}{\square\metre}} & \multicolumn{1}{c}{\teuro} &\multicolumn{1}{c}{\teuro} & \multicolumn{1}{c}{\teuro} & \multicolumn{1}{c}{\teuro} & \multicolumn{1}{c}{\teuro}  \\ \midrule
\multirow{3}{*}{A}   & 6.277,60                                                     & \multicolumn{1}{c}{/}                                & \multicolumn{1}{c}{/}          & \multicolumn{1}{c}{/}               & \multicolumn{1}{c}{/}               & \multicolumn{1}{c}{/}                                      & \multicolumn{1}{c}{/}                                              & \multicolumn{1}{c}{/} \\
                     & 12.555,20                                                    & \multicolumn{1}{c}{/}                                & \multicolumn{1}{c}{/}          & \multicolumn{1}{c}{/}               & \multicolumn{1}{c}{/}               & \multicolumn{1}{c}{/}                                      & \multicolumn{1}{c}{/}                                              & \multicolumn{1}{c}{/} \\
                     & 62.776,00                                                    & \multicolumn{1}{c}{/}                                & \multicolumn{1}{c}{/}          & \multicolumn{1}{c}{/}               & \multicolumn{1}{c}{/}               & \multicolumn{1}{c}{/}                                      & \multicolumn{1}{c}{/}                                              & \multicolumn{1}{c}{/} \\\midrule
\multirow{3}{*}{B}   & 119.604,80                                                   & 113.327,20                                           & 7,58                           & 22.740,00                           & 946,89                              & 9.468,94                                                   & -90.587,20                                                         & -103.858,26           \\
                     & 239.209,60                                                   & 226.654,40                                           & 15,17                          & 45.510,00                           & 1.895,04                            & 18.950,36                                                  & \cellcolor[HTML]{FD6864}-181.144,40                                                        & -207.704,04           \\
                     & 1.196.048,00                                                 & 1.133.272,00                                         & 75,84                          & 227.520,00                          & 9.473,93                            & 94.739,33                                                  & -905.752,00                                                        & -1.038.532,67         \\\midrule
\multirow{3}{*}{C}   & 16.850,40                                                    & 10.572,80                                            & 3,79                           & 11.370,00                           & 473,45                              & 4.734,47                                                   & 797,20                                                             & -5.838,33             \\
                     & 33.700,80                                                    & 21.145,60                                            & 7,58                           & 22.740,00                           & 946,89                              & 9.468,94                                                   & \cellcolor[HTML]{FFFC9E}1.594,40                                                           & -11.676,66            \\
                     & 168.504,00                                                   & 105.728,00                                           & 37,92                          & 113.760,00                          & 4.736,97                            & 47.369,66                                                  & 8.032,00                                                           & -58.358,34            \\\midrule
\multirow{3}{*}{D}   & 9.581,60                                                     & 3.304,00                                             & 3,79                           & 11.370,00                           & 473,45                              & 4.734,47                                                   & 8.066,00                                                           & 1.430,47              \\
                     & 19.163,20                                                    & 6.608,00                                             & 7,58                           & 22.740,00                           & 946,89                              & 9.468,94                                                   & \cellcolor[HTML]{67FD9A}16.132,00                                                          & 2.860,94              \\
                     & 95.816,00                                                    & 33.040,00                                            & 37,92                          & 113.760,00                          & 4.736,97                            & 47.369,66                                                  & 80.720,00                                                          & 14.329,66             \\ \bottomrule 
\end{tabular}
\end{table}

Nella tabella \ref{ISOvincitore} sono rappresentati i confronti dei quattro isolanti riportando i valori rispetto all'isolante di partenza.
Valori che portano a dire che l'isolante vincitore è il quarto e che permette di risparmiare \SI{16132.00}{\teuro} nel caso delle cinque villette a schiere di due piani.
Sono stati riportati -- per pura speculazione -- anche i dati relativi ad una possibile abitazione di 10 piani.
\e evidente il maggior risparmio con la soluzione D.

Nel confronto non sono stati tenuti in considerazione né il trasporto né il possibile guadagno dovuto al minor spazio occupato in cantiere dall'isolante.
Si è supposto infatti che l'origine delle fabbriche delle aziende produttrici fosse in una posizione tale da non variare significativamente il costo del trasporto. 
Non sono stati presi in considerazione neanche la manutenzione né la tempistica di posa in opera, in quanto si è supposto fossero uguali per tutte le tipologie.

\e anche vero che l'isolante B -- quello con le maggiori prestazioni -- potrebbe richiedere un viaggio in meno, ma l'enorme differenza di costo intrinseco del materiale fa sì che ciò non incida.
C'è da dire che questo isolante, proprio grazie alle sue ottime caratteristiche, viene utilizzato solitamente nei ponti termici causati ad esempio dai pilastri in calcestruzzo di una struttura a telaio, per tanto il suo costo elevato viene spalmato in una superficie ben minore di quella dell'intero edificio.
        \section{Rivestimento esterno}
\begin{table}[htbp]
\caption[Elenco dei rivestimenti esterni presi in considerazione e descrizione sintetica delle loro caratteristiche]{Elenco dei rivestimenti esterni presi in considerazione e descrizione sintetica delle loro caratteristiche. Per quelli con un asterisco $^\star$ è stata effettuata un'analisi dei prezzi.}
\label{MaterialiRIV}
\centering
\begin{tabularx}{\textwidth}{rXX}
    \toprule
        \textbf{A} & \textbf{BenessereBio intonaco $\,^\star$} & Biointonaco termo-deumidificante, antimuffa e anti condensa. 
        Adatto a tutti i tipi di muro. Traspirante ed ad alta efficienza energetica. \\\midrule
        \textbf{B} & \textbf{Intomap R1 $\,^\star$} & Intonaco di fondo su murature miste, laterizio nuovo, blocchi in calcestruzzo e cemento armato gettato.  Buona adesione, particolarmente indicato per essere applicato con intonacatrice. 
        Rasatura con Planitop 510 in calce-cemento a tessitura fine.\\\midrule
        \textbf{C} & \textbf{Powerpanel HD $\,^\star$} & Rivestimento per ambienti esterni con lastre in conglomerato cementizio, di peso ridotto, facili da lavorare e durevoli agli agenti atmosferici.\\\midrule
        \textbf{D} & \textbf{Facciata ventilata BBuilding in larice} & Costituita da pannelli premontati in stabilimento composti da una sottostruttura portante e da doghe posate orizzontalmente, completamente realizzata in legno di larice massello non impregnato.\\\midrule
        \textbf{E} & \textbf{Facciata ventilata gres porcellanato} & Sistema strutturale di rivestimento esterno degli edifici per combinare estetica, funzionalità, manutenzione ed efficienza energetica. \\\midrule
        \textbf{F} & \textbf{Facciata ventilata in lamiera Prefa} & Possono essere utilizzate sia per esterni che per interni per rivestire pareti e soffitti. Il fissaggio a scomparsa mediante un sistema ad incastro maschio - femmina garantisce un'estetica gradevole. Si ottengono facciate di facile manutenzione e all'avanguardia per molti decenni.\\
    \bottomrule
\end{tabularx}
\end{table}
\begin{landscape}
\begin{table}[p]
\caption[Piano di manutenzione del rivestimento esterno]{Piano di manutenzione del rivestimento esterno. 
\e stato considerato rispetto i 30 anni di vita utile del migliore, ovvero la facciata ventilata in lamiera Prefa.}
\label{RIV_PianoManutenzione}
\centering\scriptsize
\begin{tabular}{@{}lllllrcr@{}}
\toprule
Elem. Tecnico & Tipologia Elem. & Tipologia & Soggeto & Cadenza & Costo singolo & Ripetizioni & \multicolumn{1}{l}{Costo} \\ 
Manutenibile & Tecnico & intervento & Esecutore & (Anni) & intervento (\teuro) & \multicolumn{1}{l}{previste} & \multicolumn{1}{c}{totale (\teuro)} \\ \midrule
\multirow{4}{*}{Intonaco} & \multirow{4}{*}{BenessereBio intonaco} & Controllo generale delle parti a vista & Utente & 0,5 &   0,00 & 60 &   0,00 \\
 &  & Pulizia delle superfici & Pittore & 10 &   6.608,00 & 2 &   13.216,00 \\
 &  & Sostituzione delle parti più soggette ad usura & Muratore & 10 &   891,09 & 2 &   1.782,18 \\
 &  &  &  &  &  & TOT &   14.998,18 \\ \midrule
\multirow{4}{*}{Intonaco} & \multirow{4}{*}{Intomap R1} & Controllo generale delle parti a vista & Utente & 0,5 &   0,00 & 60 &   0,00 \\
 &  & Pulizia delle superfici & Pittore & 10 &   6.608,00 & 2 &   13.216,00 \\
 &  & Sostituzione delle parti più soggette ad usura & Muratore & 10 &   995,83 & 2 &   1.991,65 \\
 &  &  &  &  &  & TOT &   15.207,65 \\ \midrule
\multirow{4}{*}{Rivestimento lapideo} & \multirow{4}{*}{Powerpanel HD} & Controllo generale delle parti a vista & Utente & 1 &   0,00 & 30 &   0,00 \\
 &  & Pulizia delle superfici con idropulitrice & Pittore & 10 &   5.286,40 & 2 &   10.572,80 \\
 &  & Sostituzione delle parti più soggette ad usura & Pavimentista & 10 &   1.984,05 & 2 &   3.968,10 \\
 &  &  &  &  &  & TOT &   14.540,90 \\\midrule
\multirow{4}{*}{Rivestimento ligneo} & \multirow{4}{*}{Facciata ventilata BBuilding} & Controllo generale delle parti a vista & Utente & 0,5 &   0,00 & 60 &   0,00 \\
 &  & Ripristino degli strati protettivi & Pittore & 10 &   7.486,86 & 2 &   14.973,73 \\
 &  & Sostituzione delle parti più soggette ad usura & Pavimentista & 10 &   5.088,16 & 2 &   10.176,32 \\
 &  &  &  &  &  & TOT &   25.150,05 \\ \midrule
\multirow{4}{*}{Rivestimento lapideo} & \multirow{4}{*}{Facciata ventilata gres} & Controllo generale delle parti a vista & Utente & 0,5 &   0,00 & 60 &   0,00 \\
 &  & Ripristino degli strati protettivi & Pittore & 10 &   2.808,40 & 2 &   5.616,80 \\
 &  & Sostituzione delle parti più soggette ad usura & Pavimentista & 10 &   2.940,56 & 2 &   5.881,12 \\
 &  &  &  &  &  & TOT &   11.497,92 \\\midrule
\multirow{2}{*}{Rivestimento metallico} & \multirow{2}{*}{Facciata ventilata lamiera} & Controllo generale delleparti a vista & Utente & 0,5 &   0,00 & 60 &   0,00 \\
 &  &  &  &  &  & TOT &   0,00 \\ \bottomrule
\end{tabular}
\end{table}
\end{landscape}
Per quanto riguarda il rivestimento esterno, non per tutti è stato possibile trovare direttamente il prezzo unitario nel prezzario. 
Pertanto, per quelli che in tabella \ref{MaterialiRIV} riportano un asterisco, è stato necessario eseguire un'analisi dei prezzi utilizzando i dati dichiarati dai produttori. 
Questo a causa della particolare natura di quel materiale e specifico di un'azienda in particolare, o relativamente nuovo sul mercato da esser aggiunto al prezzario della Provincia di Trento.

Nel caso del rivestimento esterno è stata valutata come molto incidente la manutenzione da attuare in corso d'opera, essendo il rivestimento esposto alle intemperie e non protetto come gli altri strati. 
Non è stato preso in considerazione il tempo di posa, perché si è supposto fosse simile tra tutti.
Nemmeno il parametro dello spessore è stato considerato, perché in questo caso sono tutti ben che minimo uguali.

A causa quindi della durata del materiale è stato creato un piano manutenzione esposto nella tabella \ref{RIV_PianoManutenzione} a pagina \pageref{RIV_PianoManutenzione}.

Nella tabella \ref{RIVvincitore} è riportato il sommario tra prezzi unitari finali e la manutenzione. 
Sono riportati infine i costi finali del rivestimento, evidenziando come la soluzione con l'intonaco A sia quella più conveniente in termini di costi totali.
\begin{table}[htb]
\caption[Analisi del rivestimento esterno]{Analisi del rivestimento esterno. Somma tra i costi del materiale e i costi di manutenzione. L'ultima colonna evidenza il minor costo finale della soluzione A }
\label{RIVvincitore}
\centering\scriptsize
\begin{tabular}{@{}rrrrr@{}}
\toprule
& \multicolumn{1}{c}{Prezzo unitario} & \multicolumn{1}{c}{Costo materiale} & \multicolumn{1}{c}{Manutenzione} & \multicolumn{1}{c}{Costo}  \\ 
 & \multicolumn{1}{c}{\teuro/mq} & \multicolumn{1}{c}{\teuro} & \multicolumn{1}{c}{\teuro} & \multicolumn{1}{c}{\teuro} \\\midrule
A & 14,97 &  9.893,09 &  14.998,18 &  \cellcolor[HTML]{3FE52C}24.891,27 \\
B & 18,14 &  11.986,91 &  15.207,65 &  \cellcolor[HTML]{13AE14}27.194,56 \\
C & 46,05 &  30.429,84 &  14.540,90 &  \cellcolor[HTML]{CFE703}44.970,74 \\
D & 140,00 &  92.512,00 &  25.150,05 &  \cellcolor[HTML]{F66E51}117.662,05 \\
E & 75,00 &  49.560,00 &  11.497,92 &  \cellcolor[HTML]{FBDA59}61.057,92 \\
F & 130,00 &  85.904,00 &  0,00 &  \cellcolor[HTML]{FB813F}85.904,00 \\ \bottomrule
\end{tabular}
\end{table}

        \section{Considerazioni sul vincitore}
init
%Capitoli analisi intero edificio
    \chapter{Analisi di pareti interne e solai}
\begin{table}[htb]
\caption[Confronto dei costi dell'intera struttura]{Confronto dei costi dell'intera struttura paragonando i costi totali e i guadagni dovuti alla vendita della superficie risparmiata}
\label{STRUTConfrontoFinale}
\centering
\begin{tabular}{@{}lrrrr@{}}
\toprule
 & \multicolumn{1}{c}{Muratura portante} & \multicolumn{1}{c}{X-LAM} & \multicolumn{2}{c}{Differenza} \\
 & \multicolumn{1}{c}{\teuro} & \multicolumn{1}{c}{\teuro} & \multicolumn{2}{c}{\teuro} \\\midrule
Pareti esterne & 51.330,52 & 94.881,44 & 43.550,92 & \multirow{2}{*}{-13.089,08} \\
Guadagno par esterne & 0,00 & -56.640,00 & -56.640,00 &  \\
Pareti interne e solai & 210.892,97 & 351.564,99 & & 140.672,01   \\
 &  &  &  &  \\\midrule
Totale & 262.223,49 & 389.806,43 & & 127.582,93   \\ \bottomrule
\end{tabular}
\end{table}

%Capitoli considerazioni sui vincitore e conclusioni
    \chapter{Considerazioni sui finalisti}
Si vuole ora analizzare più del dettaglio il confronto tra la soluzione zero in muratura portante e la struttura in \xlam. 
Si è visto come l'\xlam{} sia più vantaggioso nel caso si analizzino solamente le pareti esterne, perché grazie al suo spessore minore (a parità di prestazioni) permette di risparmiare molta superficie e quindi di trarne un guadagno dalla vendita.
Questo però non è sufficiente a colmare il maggior costo intrinseco del materiale non appena si prende in considerazione l'intera struttura dell'edificio. 
Nei solai infatti, non c'è nessun guadagno dall'avere uno spessore minore, se non quello di avere un pacchetto più esile, ma ciò non si traduce in ricavi.

Si elencheranno ora delle opzioni per poter capire se abbia senso o meno, a fronte dei \SI{128000}{\teuro} in più, scegliere comunque la soluzione in \xlam. 
O se invece è preferibile optare per la muratura portante perché i benefici non sono abbastanza.
\section{Vantaggi di costruire in X-LAM}
\paragraph{Tempi di costruzione}
Si è visto nel paragrafo dedicato alle strutture delle pareti esterne, come il tempo sia stato incidente nel definire gli importi delle varie soluzioni adottate.
Non si è ancora considerato il cantiere nel suo intero insieme, considerando i tempi di attesa e di gestione.

Utilizzare un materiale che necessita di posa in opera, o uno che invece ha la sola necessità di essere assemblato, cambia radicalmente i tempi del cantiere.
Sebbene con l'\xlam{} serva una gru per la posa dei pannelli, la posa e l'indurimento dei componenti in calcestruzzo del solaio in laterocemento comporta un tempo che è dell'ordine di 1 o 2 mesi contro le 2-3 settimane necessarie con l'\xlam. 
Avere un tempo di costruzione di alcune settimane in meno comporta sia la vendita anticipata: cosa che in alcune situazioni può significare enormi guadagni per il committente, che può iniziare ad avere un ritorno economico del proprio investimento in tempi molto più rapidi.
Sia costi di gestione del cantiere e di facilità di verifica della corretta posa enormemente minori. Avendo elementi prefabbricati realizzati tramite apposite macchine a taglio numerico CNC, tutti gli elementi hanno una perfetta dimensione rispetto il progetto. Non si avranno quindi possibilità di richiesta di varianti in corso d'opera dovute ad elementi che non si incastrano a causa di qualche centimetro diverso.

Per grandi strutture è differente anche il numero di operai che servono a realizzare la struttura. Utilizzando una struttura in \xlam{} sono sufficienti tre persone per una normale unità familiare. Per ottenere lo stesso quantitativo di tempo utilizzando la muratura portante, sono necessarie almeno il doppio degli operai. Avendo così più costi di manodopera e più personale da riuscire a gestire all'interno del cantiere. 

Per riportare un esempio: il palazzo progettato da Land Lease a Melbourne, ovvero il primo edificio di dieci piani ad essere costruito in \xlam, è stato realizzato in appena 38 giorni. Comparandolo con un edificio simile costruito con telaio in calcestruzzo o con muratura portante, il tempo sale a 20 settimane \cite[39]{10storey}. 


\paragraph{X-LAM come isolante}
Nell'analisi del pacchetto isolante non si è considerato il fatto che l'X-LAM è ben più isolante rispetto la muratura portante. I due, infatti, hanno rispettivamente una conducibilità termica $\lambda$ di \SI{0.12}{} e di \SI{0.30}{W\per\metre K}.
Ciò consente di supporre che l'\xlam{} potrebbe essere considerato come contribuente alla resistenza termica della parete. 

Mantenendo una resistenza voluta di \SI{4}{\square\metre K\per W} è possibile utilizzare \SI{2}{\centi\metre} in meno di isolante e ciò comporta due cose: si risparmiano altri due centimetri lungo tutto il perimetro, potendo ricavarne dalla vendita; si ha uno spessore isolante (utilizzando il vincitore D) ridotto a \SI{8}{\centi\metre} e che ha un prezzo unitario di \SI{23.73}{\teuro} al posto dei \SI{29.00}{\teuro} iniziali.

Questo si traduce in \SI{11370}{\teuro} recuperati dalla vendita e da \SI{3482.65}{\teuro} utilizzando l'isolante meno spesso. 
Non è un risparmio molto elevato, ma nell'ipotesi che l'edificio sia di ben altre dimensioni, la situazione sarebbe diversa. 

Non solo. 
Se si considera l'intero \textit{Life Cycle Analysis} (LCA), avere molto isolante può comportare danni ambientali e di salubrità dell'ambiente interno \cite{reijnders_comprehensiveness_1999}. 
Avere un materiale che può in parte sostituirlo, sicuramente può aiutare. 

Un ulteriore considerazione va fatta sulla traspirabilità del materiale. 
Il legno infatti permette di avere un ambiente interno senza la presenza di muffe causate dall'umidità che è già presente nella malta e nell'intonaco in fase di posa \cite{sutton_introduction_nodate}.
\paragraph{X-LAM come legno}
Utilizzare un prodotto eco-sostenibile come il legno comporta innumerevoli vantaggi dal punto di vista della salute \cite{EnergyCost}. Dal punto di vista ambientale l'\xlam{} è in grado di assorbire \SI{2314}{\tonne \per \cubic\metre} di $CO_2$, portandolo ad avere un impatto ambientale nettamente inferiore rispetto gli altri materiali \cite{hammond2011inventory}.

Inoltre in termini economici si possono ottenere finanziamenti o mutui per la costruzione con determinati materiali. 
Utilizzando prodotti lignei è inoltre possibile acquisire vari punti extra nelle certificazioni energetiche quali Casa Clima o LEED. 
Certificazioni che potrebbero far aumentare non poco il valore immobiliare degli edifici costruiti. 
Un altro aspetto a vantaggio può essere l'opportunità di recuperare dei punti CAM in altri punti progettuali nei quali, magari, erano difficili da rispettare.

Con uno sguardo più ampio: utilizzare prodotti lignei aiuta l'intera catena di gestione delle foreste \cite{callegari2010production}. 
Essendo una metodologia di costruzione nata prettamente in Italia e Austria contribuisce alla valorizzazione dell'industrializzazione nel nostro territorio..
\paragraph{Materiale leggero e performante}
L'\xlam{} ha una densità circa la metà di quella della parete muraria. 
Ciononostante, con i suoi \SI{450}{\kilogram\per\cubic\metre}, riesce ad avere un rapporto resistenza-peso che è circa il doppio. 
Questo vuol dire che costruendo con l'\xlam{} si ottiene una struttura quattro volte meno pesante e con la stessa resistenza. 
Si è visto infatti come non ci siano problemi a raggiungere edifici di dieci piani \cite{10storey}.

Avere una struttura leggera e duttile ha enormi benefici dal punto di vista sismico e permette di avere minori carichi trasmessi alle fondazioni. 
Il tutto si traduce in minori costi.

Un'ultima considerazione da fare riguarda la capacità di resistenza al fuoco. 
Sebbene in legno, la parte interna non soggetta al fuoco è capace di resistere alla pari delle proprie capacità in condizioni normali. 
Pertanto creando elementi nei quali si considera questo aspetto, si ha un tempo di resistenza ben maggiore dell'acciaio presente nei solai in laterocemento (o meglio ancora dei telai in acciaio). 
Tutto questo senza l'utilizzo di vernici o materiali protettivi, e quindi senza ulteriori costi da prendere in considerazione.
%
\section{Conclusioni}
A fronte dei numerosi vantaggi che si sono qui sopra elencati in maniera per lo più discorsiva, sta ora all'investitore scegliere se spendere di più per un materiale che inizialmente costa maggiormente ma che nel corso degli anni potrebbe ripagarlo in termini di velocità di costruzione e di manutenzione generale, o se invece puntare alla scelta più economica e più standard.
Da considerare infatti sono le propensioni che ha il mercato immobiliare. 
Per il comune cittadino, compratore di quell'unità immobiliare, avere una serie di vantaggi di salute e di ricavi spalmati nel tempo ma un costo iniziale più alto rispetto alla controparte in normale muratura, fa propendere per quest'ultima perché ritiene di poco interesse i benefici della prima.

In ottica generale e non prettamente legata alla costruzione di queste cinque villette a schiera, si può dire che potrebbero esserci dei casi in cui ci sia un vantaggio economico a favore dell'\xlam{} già solo considerando i costi iniziali. 
Ne è l'esempio un palazzo di molti piani destinato all'affitto o alla vendita.
In quel caso il ricavo dalla superficie libera disponibile sarebbe molto maggiore.
Supponendo poi che l'edificio non sia ubicato a Trento ma in una località turistica nella quale il prezzo di vendita non sia di soli \SI{3000}{\teuro \per \square\metre}, i vantaggi sarebbero ben più evidenti.
Entrerebbe poi in gioco maggiormente il fattore tempo, il quale si è visto esser ben più che dimezzato in edifici di grandi dimensioni.


\vspace{1cm}
\noindent \enquote{CLT is cost-competitive because it already has thermal insulation, \omissis{} and for sure it might be a little bit more expensive in the beginning, but when you also include the maintenance costs it turns out be absulutely cost-competitive. -- \textcite{mallo_outlook_2014}}

\vspace{1cm}
Per quanto riguarda gli altri strati del pacchetto, i vincitori trovati nell'analisi sono stabili per qualsiasi tipologia di edificio.

La scelta della rifinitura esterna può comunque prendere in considerazione i vantaggi ambientali che potrebbero derivarne scegliendone una più costosa.

L'isolante D invece mostra come si possa ricavarne un buon profitto seppur in una villetta a schiera venga messo solo su due pareti, avendo le altre in comune e interne. 
Rapportando il guadagno alla singola unità immobiliare, si ha il maggior vantaggio nel caso in cui ce ne sia una per ogni piano. 
Avendo quindi una tipologia edilizia con un'abitazione singola o un palazzo in cui ogni piano sia una diversa unità, si hanno tutte le quattro le pareti esterne isolate. 
Si veda a tal proposito l'esempio con 10 piani riportato in tabella \ref{ISOvincitore}.
    %ALLEGATI IN FONDO A TUTTO:
\appendix
\renewcommand{\thechapter}{Allegati} %Toglie A,B,C delle appendici del TOC sostituendolo con "Allegati"
%Funziona solo perché c'è solo un appendice sennò non avrebbe senso
\titleformat{\chapter}%toglie A, B, C e sposta il nome a sinistra
        {\normalfont\Huge\bfseries}{}{0em}{} 
\chapter[]{Allegati} %Nell'indice non ha nome []
\begin{enumerate}
    \item Piante edificio
    \item Zonizzazione dello stato zero
    \item WBS stato zero
    \item Computo metrico estimativo struttura perimetrale
    \item Computo metrico estimativo struttura pareti interne e solai intermedi e di copertura
\end{enumerate}
%\clearpage serve perché i ref venivano sbagliati
%\phantomsection serve perchè l'xyperlink non riportata al punto giusto

%Piante edificio:
%I pdf originali sono in A3 a scala 1:50 in realtà. Ora è diventato in 1:100 A4
\clearpage
\phantomsection
\label{piante}
\includepdf[pages={1},pagecommand={\thispagestyle{plain}}]{img/PiantaTerraA3scala50.pdf}
\includepdf[pages={1},pagecommand={\thispagestyle{plain}}]{img/PiantaPrimoA3scala50.pdf}
%Zona ubicazione edificio
\clearpage
\phantomsection
\label{Edificio}
\includepdf[pages={1,2},pagecommand={\thispagestyle{plain}}]{img/Zonizzazione1.pdf}
\includepdf[pages={1},pagecommand={\thispagestyle{plain}}]{img/Zonizzazione2.pdf}
%WBS:
\clearpage
\phantomsection
\label{WBS}
\includepdf[pages={1,2},pagecommand={\thispagestyle{plain}}]{img/WBStable21.pdf}
%Conputo strutture perimetrali:
\clearpage
\phantomsection
\label{STRUTcostoMateriale}
\includepdf[pages={1-3},pagecommand={\thispagestyle{plain}}]{img/STRUTcostoMateriale.pdf}
%Conputo strutture interne e solai:
\clearpage
\phantomsection
\label{STRUTMuraturaTotaleIntESol}
\includepdf[pages={1},pagecommand={\thispagestyle{plain}}]{img/STRUTMuraturaTotaleIntESol.pdf}
%
\clearpage
\phantomsection
\label{STRUTXLaMTotaleIntESol}
\includepdf[pages={1},pagecommand={\thispagestyle{plain}}]{img/STRUTXLAMTotaleIntESol.pdf}
\end{document}